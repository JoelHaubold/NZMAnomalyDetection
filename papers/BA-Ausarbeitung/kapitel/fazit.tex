%!TeX root=../main.tex
% fazit.tex
\chapter{Fazit}
\label{chapter:fazit}

In dieser Arbeit wurden das Isolation Forest Verfahren aus \cite{liu2012isolation}, sowie das darauf basierende Robust Random Cut Forest Verfahren aus \cite{guha2016rrcfTheory} aufgearbeitet. Weiterhin wurde die Eignung dieser beiden Verfahren für die Anomalieerkennung, auf einem aus der Niederspannung stammenden Datensatzes getestet. Dabei wurde insbesondere eine Verbesserungsmöglichkeit der Erkennung von Anomalien des Random Forest Verfahrens, durch die Reduzierung der Granularität des Inputs aufgezeigt. Während der Random Forest insbesondere für die Erkennung der Sprunganomalien, aufgrund deren genauen Bedingungen, die einen anomalen Punkt stark von einem normalen Punkt abgrenzen geeignet ist, konnte dieser auch in anderen Klassen eine hohe Genauigkeit vorweisen. Diese beruht auf einer Erfolgreichen Erkennung von normalen Punkten als solche, es besteht jedoch ein Defizit bei der erfolgreichen Erkennung von Anomalien. Das Isolation Forest Verfahren konnte ohne einen genauen vorgegebenen Anteil von Anomalien in den Testsätzen, nur bedingt zwischen einer Anomalie und einem normalen Punkt unterscheiden, und mit der starken Variation des Anomalievorkommens zwischen den Stationen und Testsätzen lag dessen Performance, unter der Aufgabenstellung dieser Arbeit unter der des Random Forest Verfahrens. Dies stellt zumindestens die Erfüllung einer notwendigen Bedingung für die Sinnhaftigkeit der Anpassungen des Random Forest Verfahrens gegenüber des Isolation Forest Verfahrens dar.

Eine Anwendung eines weiteren unüberwachten Lernverfahrens, zum Vergleich mit dem Random Forest Verfahren, wie eine One-Class-Support-Vector-Machine könnte eine stärkere Aussage über die Eignung dieser, für diese spezifische Aufgabenstellung treffen.

Eine weitere Verbesserung der Klassifizierungsperformance des Random Forest Verfahren könnte sich auch durch eine Anwendung eines Stichprobenerfahrens auf den Input ergeben. Entweder durch die Ausnutzung der auf dem Einfügen eines Punktes in die RRCTs direkt folgenden Bewertung, indem dieser, falls er als Anomalie erkannt werden sollte, direkt wieder aus diesem gelöscht wird sobald die nächste Messung vorliegt, oder durch eine Verteilung der eintreffenden Punkte auf jeweils nur eine Teilmenge der Bäume im Forest, analog zu dem Isolation Forest Verfahren. Ebenfalls könnte der Effekt einer genaueren Einschränkung der Granularität des Inputs, basierend auf den positiven Ergebnis der strikten Einschränkung auf drei Nachkommastellen des Inputs des RRCF Verfahrens weiter untersucht werden. Auch eine Anpassung des Verifizierungsverfahrens der Ergebnisse, zur Eignung für die Anwendung eines Fensterverfahrens für den Input, könnte basierend auf den generell positiven Ergebnissen den ein solches Verfahren auf Zeitreihen mit sich bringt die Performance des Random Forest Verfahrens weiter verbessern.

Die verwendete Methode der Vorkalkulation der Spannungswerte für die einzelnen Anomalieklassen erweist sich im Falle der Seasonanomalie als nicht praktikabel für eine live Anwendung. Für diese Klasse wäre ein Verfahren besser, welches auf die Anomalien von den Spannungsverläufen alleine schließen kann. Ein entgegen der Aufgabenstellung überwachtes Verfahren, wie ein Replicator Neural Network, könnte sich als geeignet dafür erweisen, indem es die 5 vorliegenden Anomalieklassen über den vorliegenden Testdatensatz rein anhand des absoluten Spannungsverlaufes zu erkennen lernt. Dieses könnte über eine live Anbindung der Messgeräte zentralisiert die Spannungsverläufe auswerten, und einzelne Punkte, falls als Anomalie erkannt zu einer oder mehreren Anomalieklassen zuordnen.

Letztendlich konnte eine Eignung des Random Forest Verfahren, entsprechend einem Teil der Aufgabenstellung als Anomalieerkennungsverfahren, gezeigt werden, da durch die hohe Erfolgsquote bei der Klassifizierung von normalen Punkte als solcher, eine von dem Verfahren erkannte Anomalie, oft eine Anomalie darstellt und somit die Betrachtung einer Fachkraft begründet. Für die erfolgreiche Erkennung beinahe aller Anomalien ist das Random Forest Verfahren, zumindest in Form der in dieser Arbeit angewandten Implementation jedoch nicht in der Lage.