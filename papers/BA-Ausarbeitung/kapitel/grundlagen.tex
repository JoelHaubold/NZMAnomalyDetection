% kapitel2.tex
\chapter{Grundlagen}
\label{chapter:grundlagen}

\section{Notationen}
Die in dieser Arbeit verwendeten Notationen sind:

\begin{itemize}
\item :DDDDD
\end{itemize}

\section{Zeitreihenanomalieerkennung}

\subsection{Anomalie}
In einem gegebenen Datensatz an Punkten, wird einer dieser Punkte als Outlier bezeichnet, falls er die Kompläxität des Datensatzes überdurchschnittlich gegenüber den anderen Punkten  erhöht.

\subsection{•} 

Definition Anomalie
Definition Zeitreihe
Besondere Eigenschaften

\section{Anomalierkennung durch Random Forests}

