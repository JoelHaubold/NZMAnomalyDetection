%!TeX root=../main.tex
% tests.tex
\chapter{Tests auf Niederspannungsdaten}
\label{chapter:tests}

\section{Vorteile von RRCF} % Einleitung zu Bedarf des Datensatzes hier vor, oder ans Ende des Kapitels packen, oder in das nächste Kapitel so das dies ein reines Theorykapitel wird, oder Datensatz unabhängig beschreiben

RRCF wird zur Analyse des dieser Arbeit zugrunde legendem Datensatzes benutzt, da dass Verfahren eine Reihe von Vorteilen besitzt \cite{bartos2019rrcfImpl}:
\begin{itemize}
\item \textit{Anwendbarkeit auf Streaming-Daten}: Neue Datenpunkte können in die konstruierten Bäume eingegliedert werden ohne das diese neu aufgebaut werden müssen.
\item \textit{Geeignet für hoch dimensionale Daten}: Die angewandte Baumstruktur ist sehr geeignet für das aufnehmen von hochdimensionalen Daten. Da der Algorithmus zwischen wichtigen und unwichtigen Dimensionen unterscheiden kann, wird auch der Einfluss von solchen unwichtigen Dimensionen eingeschränkt.
\item \textit{Robust gegenüber Duplikaten}: Duplikate
\item \textit{Ausgabe in vorm einer Bewertung}: Eine Bewertende Ausgabe ist nützlich, da
\end{itemize} 