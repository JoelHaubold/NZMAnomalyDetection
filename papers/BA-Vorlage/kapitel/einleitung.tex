% einleitung.tex
\chapter{Einleitung}
\section{Motivation und Hintergrund}
\subsection{Physikalischer Hintergrund}
Eine Referenz~\cite{AggarwalV88}.
Das deutsche Verteilnetz wurde ursprünglich mit dem Ziel gebaut, den  in Großkraftwerken produzierten Strom und über das Transportnetz in die einzelnen Regionen Deutschlands transportiert wird, regional an die Endkunden (sowohl Industrie- und Gewerbekunden als auch Haushalte) zu verteilen. Das Verteilnetz ist dabei baumartig strukturiert und besteht aus der Hochspannungsebene die den Übergabepunkt des Transportnetz enthält und sich hin zur Mittelspannungsebene, Niederspannungsebene und schließlich den Endkunden verzweigt.  \\
Mit zunehmender Integration von Erneuerbaren Energien wie Wind- und PV-Anlagen in die Mittel- und Niederspannungsebene steigt auch die Dynamik in den unteren Spannungsebenen. Lastflüsse die vorher stets von oben (Hochspannung) nach unten (Mittel-, Niederspannung) gerichtet waren, kehren sich in Teilen um und können zu einer lokal höheren Auslastung des Netzes führen. Hinzu kommen neue Verbraucher wie z.B. Elektrofahrzeuge die insbesondere in den frühen Abendstunden und über die Nacht verteilt das Netz stärker belasten. \\
Um diese Effekte erkennen und analysieren zu können, müssen die Niederspannungsebene zunächst messtechnisch erfasst werden. Die Firma PPC baut ein Messgerät, welches sich in Ortsnetzstationen (Übergabepunkt von Mittel- zu Niederspannung) einbauen lässt und dort eine dreiphasige Spannungsmessung durchführen kann. Zusätzlich verfügt das Messgerät über eine Kommunikationsanbindung mit der sich die Daten abrufen und an einem zentralen Punkt aggregieren und auswerten lassen. \\
Im Rahmen dieser Arbeit soll diese Analyse auf einem begrenzten Teil der Daten automatisiert fortgeführt werden. Dazu sollen Anomalieerkennungsverfahren automatisiert Besonderheiten in den Zeitreihen aufdecken. 

\subsection{Anomalierkennung in Zeitreihen}

Über die Jahre...


\section{Aufbau der Arbeit}
Im Rahmen dieser Arbeit werden zuerst in Kapitel 2 grundlegende Begriffe der Anomalieerkennung von Zeitreihen, sowie die beiden in dieser Arbeit verwendeten Anomalieerkennungsverfahren, rrcf und one-class-svm erläutert. In Kapitel 3 wird auf die Implementierung der beiden Verfahren eingegangen, und in Kapitel 4 werden diese beiden Verfahren anhand eines Testdatensatzes miteinander verglichen. 
