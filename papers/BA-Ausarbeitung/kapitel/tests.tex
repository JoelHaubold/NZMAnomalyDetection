%!TeX root=../main.tex
% tests.tex
\chapter{Tests auf Niederspannungsdaten}
\label{chapter:tests}

Im Rahmen dieser Arbeit wurde die Performance von zwei Anomalieerkennungsverfahren auf dem ihr zugrundeliegendem Testdatensatzes beurteilt. In diesem Kapitel wird nun zuerst auf die Eigenschaften des Datensatzes eingegangen, und darauf auf die Eignung der angewandten Anomalieerkennungsverfahren für diesen, sowie auf die Details ihrer jeweiligen Implementierung. 

\section{Aufmachung der Testdaten} \label{sec:ans}

Der Testdatensatz wurde im Jahr 2018 von in 17 unterschiedlichen Stellen des deutschen Niederspannungsnetz angebrachten Messstationen aufgezeichnet. Bemessen wurde dabei die absolute Spannung aller drei Stromphasen in Abständen von 9.5 Sekunden, wobei jede Phase zeitgleich bemessen wurde. Je nach Station bilden die aufgenommenen Daten einen Zeitraum von mindestens 3 bis zu maximal 8 Monaten ab. Insgesamt enthält der Datensatz über alle Phasen aller Stationen 66 Millionen Punkte, welche sich in wie in Tabelle \ref{tab:ppc} dargestellt aufteilen. 

\begin{table}[ht]
\caption{Der Testdatensatz aufgeschlüsselt nach den Stationen und der Anzahl an Punkten aller drei Phasen der Station, welche den jeweiligen Anomalieklassen zugehören. Die letzten beiden Reihen stellen die Gesamtgröße der Anomalieklassen in dem Datensatz dar, sowie den prozentualen Anteil den diese insgesamt an dem Datensatz haben}
\centering
\begin{tabular}{lrlllll}
\toprule
{} &  Punkte &    Sprunga. &      Zeita. &    Phasena. &   Saisona. &  Stationsa. \\
\textbf{Station} &              &             &             &             &            &             \\
\midrule
\textbf{4352   } &      3798525 &        2853 &        6723 &          66 &       4085 &          98 \\
\textbf{0928   } &      4854720 &        4377 &          84 &          12 &       5455 &        1300 \\
\textbf{0120   } &      5521035 &        6867 &          54 &           0 &          0 &         782 \\
\textbf{0691   } &      1974597 &       10563 &        4662 &       13560 &       3647 &        9205 \\
\textbf{4366   } &      4814937 &        4497 &         129 &           9 &       5474 &        1158 \\
\textbf{0942   } &      4032360 &        4965 &        8640 &           0 &       6037 &          94 \\
\textbf{4609   } &      4365249 &       23865 &       10350 &       39414 &      17326 &        1547 \\
\textbf{0595   } &      4276122 &        8310 &       12201 &           0 &         10 &         473 \\
\textbf{4623   } &      2254374 &        5427 &        3300 &           0 &          0 &          81 \\
\textbf{0888   } &      4864896 &        4677 &          99 &         375 &       4791 &        1382 \\
\textbf{0637   } &       946380 &        3513 &        8346 &        6204 &       1689 &        3059 \\
\textbf{0993   } &      5303775 &        7971 &         909 &        5589 &       1839 &       34259 \\
\textbf{3723   } &      5767935 &        6999 &          57 &           0 &          0 &        2241 \\
\textbf{4367   } &      4863876 &        4461 &          84 &         327 &       4930 &        1369 \\
\textbf{1035   } &      4061799 &        5403 &        9180 &         564 &       2493 &        7790 \\
\textbf{1145   } &      2194560 &        2478 &         657 &           0 &       3724 &         177 \\
\textbf{1146   } &      2156118 &        2937 &        1569 &          48 &       4604 &        1062 \\
\midrule
\textbf{gesamt } &     66051258 &      110163 &       67044 &       66168 &      66104 &       66077 \\
\textbf{anteil } &            1 &  0.0016678 &  0.001015 &  0.0010018 &  0.001 &  0.0010004 \\
\bottomrule
\end{tabular}
\label{tab:ppc}
\end{table}


Die absoluten Spannungswerte bewegen sich in einem Bereich von 182 V bis 236 V. Über den Gesamtmesszeitraum ergeben sich dabei starke saisonale Unterschiede, unter anderem Abhängig von der Jahreszeit und der Tagesart. %\ref{Spannung über Monat}

Die Punkte des Testdatensatzes wurden in der Nachbearbeitung maschinell zu bis zu 5 Anomalieklassen zugeordnet. Diese sind:
\begin{enumerate}
\item \textbf{Sprunganomalien:} Punkte direkt nach einer Trafostufung, also eine drei Punkte Kombination einer Messung deren jeweilige Spannungen  in der jeweilig gleichen Phase entweder jeweils ungewöhnlich größer oder jeweils ungewöhnlich kleiner sind, als die 3 Spannungen der 3 Punkte Kombination der vorherigen Messung der Messstation.
\item \textbf{Zeitanomalien} Punkte direkt nach einer Messlücke, also Punkte deren Zeitpunkt weit länger als die üblichen 10 Sekunden hinter dem Zeitpunkt ihres Vorgängerpunkts liegt.
\item \textbf{Phasenanomalien:} Punkte welche sich von den jeweiligen Punkten der anderen beiden Phasen absetzen, also Punkte deren Spannung stark von der Spannung von mindestens einer Spannung der beiden zugleich aufgenommenen Punkte der beiden anderen Phasen unterscheidet.
\item \textbf{Saisonanomalien:} Punkte die mit dem saisonalen Trend der Zeitreihe brechen, also Punkte deren Spannungswerte sich stark von den Werten vorherigen Punkte unterscheiden, welche zu einer ähnlichen Uhrzeit und in der gleichen Tagesart gemessen wurden. Dabei wurde zwischen Werktagen und der Kombination aus Feier- und Wochenendtagen unterschieden.
\item \textbf{Stationsanomalien:} Punkte welche gegen den Trend des durchschnittlichen Verlaufs aller Zeitreihen verstoßen, also Punkte deren Spannung sich stark von dem Durchschnitt der Spannung der Punkte aller anderen Stationen unterscheiden.
\end{enumerate}
Die Anomalieklassen sind nicht exklusiv, jeder Punkt kann in mehreren Anomalieklassen sein. 

Während die Zeitreihen jeder Messtation ähnliche Verhaltensweisen aufweist, treten die vorhandenen Anomalien je nach Zeitreihe in jeweils unterschiedlicher Stärke und Frequenz auf. Die Daten sind jeweils punktweise gelabeled, es zeichnen sich allerdings punktübergreifende Muster für jede Anomalieklasse ab:
\begin{itemize}
\item Aufgrund der Definition von Sprunganomalien, nach welcher jeder Punkt in einer Messung ein bestimmtes Verhalten gegenüber dem Punkt, der jeweilig gleichen Phase der vorherigen Messung haben muss, damit diese Punkte als anomal gekennzeichnet sind, sind in einer Messung entweder alle Punkte eine Sprunganomalie oder keiner von ihnen. Analog dazu sind entweder alle Punkte einer Messung als Phasenanomalie gekennzeichnet oder keine. 
\item Ähnlich dazu treten Zeitanomalien fast ausschließlich in dreier Paaren von Punkten auf, welche eine Messung einer Station darstellen, da die ihnen zugrundeliegenden Zeitlücken, fast immer Messlücken einer Messstation entsprechen und so für jede Phase einer Station sich zeitliche Lücken bilden.
\item Saison-, Phasen- und Stationsanomalien weisen ein stark geklustertes Verhalten auf, wo der Großteil der ihnen zugehörigen Punkte direkt aufeinander folgen, da in diesem Bereich der Spannungsverlauf einer Phase nach dem zugehörigen Anomaliekriterium wesentlich höher oder niedriger als erwartet ist. 
\end{itemize} 

\subsection{Eignung der Daten für überwachte und unüberwachte Anomalieerkennung}
Während die Verhaltensformen der Anomalieklassen sich überwachten Lernen anbieten können, wurde sich in dieser Arbeit dennoch für zwei unüberwachte Verfahren entschieden. Wie in Tabelle \ref{tab:ppc} zu sehen ist, sind die Anomalieklassen sehr gering vertreten, wodurch sich eine Knappheit an Daten ergibt, mithilfe derer ein überwachtes Anomalieerkennungsverfahren die jeweiligen Anomalieklassen lernen könnte. Weiterhin entsprechen die Anomalien immer einer in einem bestimmten Kontext ungewöhnlich höheren oder niedrigeren Spannung als gewöhnlich, weshalb ein unüberwachtes Verfahren, diese über Bestimmung der jeweiligen Häufigkeit der eingegebenen Punkte als solche klassifizieren kann.

\subsection{Benötigte Eigenschaften eines Anomalieerkennungsverfahrens}

Es ergeben sich drei weitere Eigenschaften für die die gewählten Anomalieerkennungsverfahren geeignet sein müssen:

\begin{enumerate}
\item Aufgrund der oben beschriebenen Tendenz mancher Anomalien in Gruppen zu klustern muss ein auf dem Datensatz angewendetes Anomalieerkennungsverfahren robust gegenüber Duplikaten seien.
\item Da die Daten über die Messstationen live erfasst werden, sollte das Anomalieerkennungsverfahren in der Lage dazu seien, seinen Input als Stream zu empfangen. Weiterhin muss das Verfahren sich an Änderung des durchschnittlichen Spannungswerts anpassen können, möglichst bevor wegen dieser Änderungen eigentliche Inliner als Anomalien klassifiziert werden.
\item Aufgrund der numerischen Klassifizierungskriterien der Anomalien, also dem Fehlen einer klaren Abgrenzung zwischen Inlinern und Anomalien, muss das Verfahren in der Lage sein die Grenze zwischen diesen zu approximieren. Im Falle eines unüberwachten Verfahrens ohne das es sich diesen anlernen kann.
\end{enumerate}

\subsubsection{Aufbau der Testsätze}

Aufgrund des hohen Umfangs der Testdaten wurden die Tests auf einem Subset der RRCF Daten ausgeführt, diese waren jeweils definiert über:
\begin{itemize}
\item Die Station der zu testenden Zeitreihe
\item Der Startzeitpunkt des zu testenden Zeitfensters
\item Die zu testende Phase
\item Die zu testende Anomalieklasse
\end{itemize}
Zur Erkennung jeder Anomalieklassen wird die absolute Spannung jeweils den entsprechenden Bedingungen der Klasse, wie in Sektion \ref{sec:ans} umgerechnet, beispielsweise wird für die Phasenanomlien aus den 3 gemessenen absoluten Spannungswerten die größte Differenz als Input verwendet. Die dazu nötigen Operationen sind für alle Anomalieklassen, bis auf die der Seasonanomalien effizient berechenbar und somit live ausführbar.
Die Gesamtheit der Testsätze wurde representativ über die Daten ausgewählt. Dabei wurde ein Fokus auf Testsätze, welche anomale Abschnitte enthalten gelegt. Testsätze, welche nur aus Inlinern bestehen wurden stellenweise hinzugefügt um die Klassifizierung von Inlinern als solche zu überprüfen. 

\section{Testen des RRCF Verfahrens}

RRCF als unüberwachtes Anomalieerkennungsverfahren eignet sich zur Analyse des dieser Arbeit zugrunde legendem Datensatzes\cite{bartos2019rrcfImpl}:
\begin{itemize}
\item \textit{Robust gegenüber Duplikaten}: Da der RRCF seine Anomalieeinschätzung in Form des CoDisps gibt, welches per Konzept Robust gegenüber Duplikaten ist, ist das Verfahren in der Lage auch mehrere sich nur schwach unterscheidende anomale Punkte in den  Bäumen als solche zu klassifizieren.  
\item \textit{Anwendbarkeit auf Streaming-Daten}: Neue Datenpunkte können in die konstruierten Bäume eingegliedert werden ohne das diese neu aufgebaut werden müssen.
\item \textit{Anpassung an Änderungen im Datensatz}: Da jeder RRCT eine endliche Anzahl an Punkten enthält, muss mit dem Einfügen von neuen Punkten in den RRCT, das Löschen von alten Punkten einhergehen. So kann das was der RRCT als Inliner klassifiziern würde, an das neuer normal angepasst werden
%\item \textit{Geeignet für hoch-dimensionale Daten}: Die angewandte Baumstruktur ist sehr geeignet für das aufnehmen von hochdimensionalen Daten. Da der Algorithmus zwischen wichtigen und unwichtigen Dimensionen unterscheiden kann, wird auch der Einfluss von solchen unwichtigen Dimensionen eingeschränkt.
\item \textit{Ausgabe in vorm einer Bewertung}: Für das RRCF Verfahren muss ein Grenzwert ermittelt werden, um den Codisp der jedem Punkt zugeordnet wird zu der binären Klassifizierung zwischen Inliner und Anomalie zu transformieren. So kann die tatsächliche Grenze der vorhandenen Anomalielabel ermittelt werden.  
\end{itemize} 

Ein weiterer Vorteil des RRCF Verfahrens, die effiziente Handhabung von hochdimensionalen Daten, wird hier nicht benutzt, öffnet aber weitere Alternativen zu der Handhabung des Datensatzes. %To the end with sampling

\subsection{Implementierung der Tests}

Über die Tests sollen die Parameter für den auf den Testdaten leistungsfähigsten RRCF gefunden werden. Zur Ermittlung der Leistungsfähigkeit wird dabei der MCC verwendet, um eine Balance zwischen den richtigen Klassifizierung von Inlinern und Anomalien zu finden. Ein F-Maß wurde ausgeschlossen, da das Verfahren in der Lage sein muss sehr viele normale Datenpunkte als solche zu klassifizieren, was in einem F-Maß nicht bewertet wird. Die Genauigkeit der Ergebnisse ist ebenfalls eine unzureichende Metrik, aufgrund des geringen Anteils von Anomalien im Datensatz, wodurch bei einer Bewertung über Genauigkeit, die Klassifizierung von Anomalien nebensächlich wäre

Da in der Praxis der RRCF live auf den von den Messstationen aufgenommen Daten laufen soll, ist es Ziel der Tests diese Situation über Ausschnitte der Daten zu simulieren. Dazu wird wie folgt vorgegangen:


\subsubsection{Ablauf der Testläufe}

Jeder Testlauf testet, auf einem Testsatz die Leistungsfähigkeit einer Kombination der folgenden drei Parameter:
\begin{itemize}
\item \textbf{Baumgröße ($ts$):} Die Anzahl an Punkten die in jeden RRCT des RRCFs passen
\item \textbf{Baumanzahl ($nt$):} Die Anzahl der RRCTs in dem konstruierten RRCF
\item \textbf{Fenstergröße:} Die Größe der Fensterabschnitte, welche die Punkte aus denen die RRCTs gebaut werden ausmachen
\end{itemize}
Der Testlauf erfolgt in drei Schritten:
 
\paragraph{Schritt 1: Simulation der Praxis} Es wird ein RRCF enstprechend der Parameter des Testlaufes erzeugt, um den in der Praxis bereits vorhanden, über die vorherig gestreamten Daten konstruierten RRCF zu simulieren. Dazu werden aus den letzten $ts$ Punkten (wobei ein Punkt je nach Fenstergröße entweder ein alleinstehender Wert oder eine Reihe von Werten ist) vor dem von dem Testlauf definierten Startpunkt, $nt$ Bäume konstruiert. Es genügt die einzelnen RRCTs nach Definition \ref{def:rrcfdef1} zu konstruieren, anstatt die Punkte einzeln in die Bäume einzufügen, da es so nach Theorem \ref{theo:wahrsch} keinen Unterschied in den erwarteten Bäumen gibt. 

\paragraph{Schritt 2: Streaming des Testsatzes} Der Testsatz wird angefangen mit dem definierten Startpunkt durch jeden RRCT des RRCF gestreamed. Da die Größe jedes RRCTs nach Schritt 1 der definierten Baumgröße entspricht, muss mit jedem eingefügten Punkt ein Punkt aus dem RRCF entfernt werden. In den Testläufen wurde dabei immer der älteste Punkt, also der Punkt welcher am frühesten gemessen wurde gewählt. Mit dem Einfügen von jedem Punkt wird das CoDisp zu diesem vom RRCF berechnet und abgespeichert.

\paragraph{Schritt 3: Auswertung der Ergebnisse} Basierend auf den generierten CoDisp Werten wird darauf der bestmögliche MCC über alle möglichen CoDisp-Grenzwerte, für ab wann ein Punkt als Anomalie klassifiziert wird ermittelt, wobei sich auf die Anomalielabels des Datensatzes bezogen wird, ob ein Punkt von dem RRCF richtig klassifiziert wurde. Daraufhin wird für den ermittelten optimalen Grenzwert die Accuracy bestimmt, um eine Vergleichsmetrik mit Testabschnitten zu haben, welche ausschließlich Inliner enthalten, da der MCC, wie in Sektion \ref{sec:performance} beschrieben dort nicht anwendbar ist.

\subsubsection{Ergebnisse}

Die ersten Testläufe, wurden gebündelt über alle Anomalieklassen ausgeführt, um den Effekt der drei Parameter der Testläufe beurteilen zu können. Dabei stach hervor, dass wie in Tabelle \ref{tab:shingle} dargestellt, die Einführung der Methodik, den Input des RRCF Verfahrens in Fenstern einzuführen, zu einer Performanceverschlechterung führte. Die Ursache dafür ist auf das punktuelle Labeling der Anomalieklassen zurückzuführen. Ein als Anomalie gelabelter Punkt des Datensatzes, ist in einer Fenstergröße von drei, beziehungsweise von fünf, drei beziehungsweise fünf Mal im Input enthalten. Bei der Auswertung der Einschätzung des RRCFs wird allerdings nur eine Klassifizierung als Anomalie genau des Fensters, welches den anomalen Punkt enthält als richtig bewertet.
Weiterhin zeichnet sich mit einer Fenstergröße höher eins ein starker Anstieg der Laufzeit des Verfahrens ab, da die Dimensionalität des Inputs erhöht wird.
\begin{table}[ht]
\caption{Der Effekt des Fenstergrößen Parameters über 150 Testsätze. Dargestellt ist die Dauer des Testlaufes sowie der durchschnittliche MCC, die durchschnittliche Dauer des Testlaufes in Minuten und die durchschnittliche Genauigkeit. Die Parameter für die Baumgröße und die Anzahl an Bäumen liefen von 10000 bis 30000, beziehungsweise von 100 bis 300 in Schritten von 5000 beziehungsweise 50.}
\centering
\begin{tabular}{lrrr}
\toprule
{} &       MCC &  Testlauf Dauer &  Genauigkeit \\
\textbf{Fenstergröße} &           &                 &              \\
\midrule
\textbf{1           } &  0.398729 &      217.164686 &     0.989065 \\
\textbf{3           } &  0.381215 &      455.646902 &     0.980104 \\
\textbf{5           } &  0.365391 &      467.922607 &     0.984984 \\
\bottomrule
\end{tabular}
\label{tab:shingle}
\end{table}

Weiterhin hat sich herausgestellt das die Reduzierung der Granularität des Inputs des Verfahrens zu einer wesentlichen Verbesserung, sowohl von dessen Genauigkeit, als auch dessen Laufzeit führte. Diese Verringerung der Granularität wurde über eine Begrenzung der Nachkommastellen des Inputs erreicht und ihr Effekt ist in Tabelle \ref{tab:nksphase} beispielhaft dargestellt.
\begin{table}[ht]
\caption{Der Effekt einer Verringerung der Nachkommastellen, angegeben als $NK$ des Inputs auf den MCC und der Testlaufdauer in Minuten. Dargestellt sind jeweils 24 Testläufe über 8 Testsätze mit 3 Wiederholungen, pro Baumgröße und Nachkommastelle. Die Fenstergröße von jedem Testlauf war 1 und die Anzahl der Bäume betrug 150.}
\centering
\begin{tabular}{lrrrrrr}
\toprule
{} &       MCC NK:3 &          Testlauf Dauer NK:3 &  MCC NK:7 &  Testlauf Dauer NK:7  \\
\textbf{Baumgröße} &                       &              &              &                 \\
\midrule
\textbf{10000    } &  0.559057 &    44.494513 &          NaN &         NaN \\
\textbf{15000    } &  0.562351 &    67.811954 &          NaN &       NaN \\
\textbf{20000    } &  0.524942 &   102.143578 &     0.689383 &     6.666684 \\
\textbf{25000    } &  0.514378 &   141.698940 &     0.696069 &     6.896325 \\
\textbf{30000    } &       NaN &             NaN &     0.741360 &     7.002047 \\
\textbf{35000    } &       NaN &             NaN &     0.725485 &    7.646865 \\
\textbf{40000    } &       NaN &             NaN &     0.728822 &    7.896554 \\
\textbf{45000    } &       NaN &           NaN &     0.733467 &      8.299229 \\
\bottomrule
\end{tabular}
\label{tab:nksphase}
\end{table}
Die Verbesserung der Laufzeit lässt sich darauf zurückführen, dass mit einer verringerten Granularität mehr Duplikate im input vorhanden sind, und somit bei der Einführung von weniger Punkten in die RRCTs neue Partitionen vonnöten sind. Ebenfalls sorgt die niedrigere Granularität für ein Zusammenklumpen der normalen Punkte, während die Anomalien durch ihre distinkten Werte, weiterhin leicht isolierbar bleiben, was zu der Verbesserung des MCCs führt, diese Verhinderung der richtigen Klassifizierung von Anomalien, durch zu viele normale leicht unterschiedliche Punkte führt dazu das die optimale Baumgröße steigt, da die dazukommenden normalen Punkte, welche kein Duplikat eines bisherigen Punktes sind sinkt.
\begin{table}[ht]
\caption{Die Testläufe zur Bestimmung der besten Baumgröße für jede Anomalieklasse, gemessen anhand des durchschnittlichen MCCs jedes Testlaufes. Jede Baumgröße wurde dabei über 24 Testläufe, mit 8 Testsätzen und 3 Wiederholungen getestet. Die Fenstergröße von jedem Testlauf war 1 und die Anzahl der Bäume betrug 150. Hervorgehoben ist pro Spalte jeweils die Baumgröße mit dem besten Ergebnis.}
\centering
\begin{tabular}{lrrrrr}
\toprule
{} &    Zeita. &   Phasena. &  Sprunga. &  Stationsa. &  Seasona. \\
\textbf{Baumgröße} &           &           &           &             &           \\
\midrule
\textbf{20000    } &       NaN &  0.603210 &       NaN &    0.536278 &  0.671732 \\
\textbf{25000    } &  0.632683 &  0.609060 &  0.904462 &    0.570382 &  0.713191 \\
\textbf{30000    } &  0.635779 &  0.648690 &  0.914529 &    0.634900 &  0.759553 \\
\textbf{35000    } &  0.642542 &  0.634799 &  0.922088 &    0.657656 &  0.767432 \\
\textbf{40000    } &  0.653790 &  \textbf{0.637720} &  0.949432 &    0.664821 &  0.773245 \\
\textbf{45000    } &  0.660279 &  0.631784 &  0.938342 &    0.673285 &  0.780158 \\
\textbf{50000    } &  \textbf{0.654150} &       NaN &  \textbf{0.947933} &    \textbf{0.687166} &  0.784906 \\
\textbf{55000    } &  0.653438 &       NaN &  0.935317 &    0.683527 &  \textbf{0.792754} \\
\textbf{60000    } &  0.650126 &       NaN &       NaN &         NaN &  0.784951 \\
\bottomrule
\end{tabular}
\label{tab:ts}
\end{table}

\begin{table}[ht]
\caption{Die Testläufe zur Bestimmung der besten Anzahl an Bäumen für jede Anomalieklasse, gemessen anhand des durchschnittlichen MCCs jedes Testlaufes. Jede Anzahl wurde dabei über 24 Testläufe, mit 8 Testsätzen und 3 Wiederholungen getestet. Die Fenstergröße von jedem Testlauf war 1 und die Größe der Bäume entsprach der besten Größe der vorherigen Testläufe zur Bestimmung dieser. Hervorgehoben ist pro Spalte jeweils die Anzahl mit dem besten Ergebnis.}
\centering
\begin{tabular}{lrrrrr}
\toprule
{} & Zeita. &  Phasena. &  Sprunga. &  Stationsa. &  Seasona. \\
\textbf{Anzahl Bäume} &        &           &           &             &           \\
\midrule
\textbf{100         } &   0.653942 &  0.630615 &  0.927817 &    0.679372 &  0.770709 \\
\textbf{150         } &   \textbf{0.655839} &  0.633044 &  0.935337 &    0.684770 &  \textbf{0.809800} \\
\textbf{200         } &   0.655421 &  0.630866 &  0.940764 &    0.686961 &  0.803877 \\
\textbf{250         } &   0.655630 &  0.628403 &  0.942023 &    0.689922 &  0.780190 \\
\textbf{300         } &   0.654832 &  \textbf{0.634816} &  \textbf{0.943043} &    \textbf{0.690939} &       NaN \\
\textbf{350         } &   0.654610 &  0.633528 &  0.936555 &    0.689280 &       NaN \\
\bottomrule
\end{tabular}
\label{tab:nt}
\end{table}
Die Testläufe zur Bestimmung der optimalen Klassifizierung wurden getrennt durchgeführt, um Unterschiede in der Aufmachung des jeweiligen Datenstroms auszunutzen. Tabelle \ref{tab:ts} stellt dabei die Ergebnisse der Suche nach der optimalen Baumgröße dar, Tabelle \ref{tab:nt} die darauffolgende Suche nach der optimalen Anzahl an Bäumen. Auffällig ist die besonders starke Performance bei der Klassifizierung der Sprunganomalien, welche sich aus dem, im Vergleich zu den anderen Anomalieklassen besonders spezifischen Kriterien ergibt die durch die Vorberechnung des Inputs leicht erkennbar ist. Während die Baumgröße nur einen logarithmischen Einfluss auf die Laufzeit der Testläufe hat, so hat die Anzahl der Bäume einen linearen Einfluss, da die Anzahl der Operationen die mit jeder Einfügung eines Punktes kommen linear steigen. Da die Verbesserung des MCCs über diese Parameter ab einer bestimmten Größe nur noch minimal ist, wird für die Anzahl der Bäume, für jede Anomalieklasse 150 als optimal gewertet, da die Laufzeit des Algorithmus ansonsten zu lang ist. Aufgrund der minimalen Zunahme der Laufzeit bei wachsender Baumgröße wird für diese die mit dem besten Ergebnis als optimal gewertet.
Somit gibt sich für die Performance des RRCF Verfahren aufgeschlüsselt nach den Anomalieklassen die Tabelle \ref{tab:rrcfres}. Anzumerken ist die hohe Genauigkeit der meisten Anomalieklassen, welche zusammen mit dem MCC darauf schließen lässt, dass das Verfahren zwar erfolgreich die normalen Punkte als solche klassifiziert und somit seine hohe Genauigkeit erzeugt, aber einige Anomalien fälschlicherweise nicht als solche erkennt.
%\begin{itemize}
%\item \textbf{Zeitanomalien:} Ein MCC von 0.6558
%\item \textbf{Phasenanomalien:} Ein MCC von 0.6348
%\item \textbf{Sprunganomalien:} Ein MCC von 0.943
%\item \textbf{Stationsanomalien:} Ein MCC von 0.6909
%\item \textbf{Seasonanomalien:} Ein MCC von 0.8098
%\end{itemize}
\begin{table}
\caption{Der, über alle 24 mit den optimalen Testparametern ausgeführten Testläufen durchschnittliche MCC, sowie die durchschnittliche Genauigkeit jeder Anomalieklasse}
\centering
\begin{tabular}{lrr}
\toprule
{}&MCC&Genauigkeit\\
\textbf{Anomalieklasse} &  &\\
\midrule
\textbf{Zeitanomalien} & 0.6558 & 0.996457 \\
\textbf{Phasenanomalien} & 0.6348 & 0.994447 \\
\textbf{Sprunganomalien} & 0.943 &  0.999650 \\
\textbf{Stationsanomalien} & 0.6909&   0.990624\\
\textbf{Seasonanomalien} & 0.8098 & 0.989687\\
\bottomrule
\end{tabular}
\label{tab:rrcfres}
\end{table}



\section{Testen des IF Verfahrens}

Die Tests des IF-Verfahren, laufen analog zu den RRCF-Tests ab, was ihre Bewertung über den MCC, sowie dem Aufbau der Teststrecken angeht. Da der Isolation Forest allerdings nicht für Streaming Inputs geeignet ist vereinfacht sich der Testablauf von diesem:
\paragraph{Schritt 1: Konstruktion des Forests}
Der iForest wird über den selben Testsätzen, wie die für das Testen des RRCF Verfahren benutzt wurden konstruiert, dabei wird ein unüberwachter Parametertrainierer benutzt um für diese die beste Kombination zu finden. Um zu überprüfen ob die Bereitstellung von Streaming Eigenschaften, welche das RRCF Verfahren über seine Abänderungen des iForests Verfahren erreicht, eine Verbesserung über der Analyse des gesamten Datensatzes in einem Forest darstellt, wird zur Konstruktion des iForests
\paragraph{Schritt 2: Auswertung der Ergebnisse}
Der iForest kalkuliert eine Abschätzung für jeden Punkt, ob dieser für ihn eine Anomalie darstellt oder nicht. Diese Abschätzung wird mit den tatsächlichen Labels verglichen, woraus der MCC des Ergebnisses kalkuliert wird

\subsection{Ergebnisse} 

Die Ergebnisse de iForest Verfahrens sind in Tabelle \ref{ifres} dargestellt. Nur unter Zuhilfenahme des Parametertrainers unterschätzt das iForest Verfahren den Anteil, den nicht anomale Punkte an dem Testdatensattz haben. Daher klassifiziert es viele normale Punkte als Anomalien, worunter der MCC entsprechend leidet. Als Hilfestellung wirde in einer zweiten Runde von Testläufen der Anteil den Anomalien im gesamten Testdatensatz über aller Stationen haben übergeben, welcher aufgrund der Größe des Datensatzes als angemessen konstant angenommen wird. Unter dieser Hilfestellung steigt die Klassifizierungsperformance des iForests merklich, leidet allerdings immer noch unter Testabschnitten oder ganzen Stationen, auf denen Anomalien der jeweiligen Klasse über oder unter vertreten sind. Die angewandte Implementierung des iForest Verfahrens hat nicht die selben Probleme wie das RRCF Verfahren mit einer hohen Granularität des Inputs, weder in seiner Laufzeit noch in seiner Performance. Daher wird auf eine Begrenzung der Nachkommastellen verzichtet.

\begin{table}
\caption{Der je Anomalieklasse über 16 ausgeführte Testläufe, über je acht Testsätze mit 1 Wiederholung durchschnittliche MCC, sowie der MCC unter übergabe des angegeben Grenzwertes}
\centering
\begin{tabular}{lrrr}
\toprule
{}&MCC&Anteil Anomalien&MCC mit Anteil\\
\textbf{Anomalieklasse} & &\\
\midrule
\textbf{Zeitanomalien}     & 0.147507 &0.001& 0.333837 \\
\textbf{Phasenanomalien}   & 0.088330 &0.001& 0.618359 \\
\textbf{Sprunganomalien}   & 0.180632 &0.00167& 0.342480 \\
\textbf{Stationsanomalien} & 0.204155 &0.001& 0.203088\\
\textbf{Seasonanomalien}   & 0.106621 &0.001& 0.669223\\
\bottomrule
\end{tabular}
\label{ifres}
\end{table}
%Abweichungen, mcc vs accuracy, ergebnisse vs anomalie spezifische ergebnisse
%phase = 0.14750792503444723 0.3338373086485978
%trafo = 0.08833066458127181 0.718359376599812
%time = 0.1806321374521998 0.342480058655457
%station = 0.20415526519894023 0.20308829189867555
%seas = 0.1066211871334278 0.6692231027794284
