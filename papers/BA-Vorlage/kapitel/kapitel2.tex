% kapitel2.tex
\chapter{Grundlagen}
\label{chapter:kap2}

\section{Definitionen}

Zeitreihen, Anomalien,...

\section{Datensatz}

Der dieser Arbeit zugrundeliegende Datensatz besteht aus 66 Millionen Spannungswerten. Diese teilen sich auf 17 dreiphasigen Zeitreihen, die sich über einen Zeitraum von bis zu 7 Monaten erstrecken und mit einer Auflösung von einem Messwert pro  10 Sekunden pro Phase
erhoben wurden. Die Zeitreihen zeichnen sich alle durch unterschiedliche Volatilität, sowie unterschiedlich frequenten Messlücken aus. 
\\
Der Datensatz enthält 5 Klassen von Anomalien:
\begin{itemize} % [noitemsep]
\renewcommand\labelitemi{--}
\item Punkte deren Spannung stark von ihrem Vorgänger abweicht
\item Punkte mit hohem zeitlichen Abstand von ihren Nachbarpunkten
\item Punkte deren Spannung stark von der Spannung der anderen beiden Phasen abweicht
\item Punkte deren Spannung stark von dem momentanen Spannungsdurchschnitt aller Stationen abweicht
\item Punkte deren Spannung sich von der saisonal erwarteten Spannung unterscheidet
\end{itemize}
Alle Anomalien im Datensatz sind als solche markiert. 

\section{Robust Random Cut Forests}

Robus Random Cut Forests (RRCF) sind...

\section{Verfahren 2}
