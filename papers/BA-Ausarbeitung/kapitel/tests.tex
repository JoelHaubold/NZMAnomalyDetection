%!TeX root=../main.tex
% tests.tex
\chapter{Tests auf Niederspannungsdaten}
\label{chapter:tests}

Im Rahmen dieser Arbeit wurde die Performance von zwei Anomalieerkennungsverfahren auf dem ihr zugrundeliegnedem PPC-Datensatz beurteilt. In diesem Kapitel wird nun zuerst auf die Eigenschaften des Datensatzes eingegangen, und darauf auf die Eignung der angewanten Anomalieerkennungsverfahren für diesen, sowie auf die Details ihrer jeweiligen Implementierung. 

\section{Aufmachung der Testdaten}

Der PPC-Datensatz wurde im Jahr 2018 von in unterschiedlichen Stellen des deutschen Niederspannungsnetz angebrachten Messstationen aufgezeichnet. Bemessen wurde dabei die absolute Spannung aller drei Stromphasen, im Normalfall in Abständen von bis zu 10 Sekunden. Je nach Station gab es einen Bemessungszeitraum von mindestens 4 bis zu maximal 10 Monaten.

Eilnzelne Punkte des PPC-Datensatzes wurden in der Nachbearbeitung zu bis zu 6 Anomalieklassen zugeordnet. Diese sind:

\begin{enumerate}
\item 
\item 
\item 
\item 
\item 
\item 
\end{enumerate}

Während die Zeitreihen jeder Messtation ähnliche Verhaltensweisen aufweist, treten die vorhandenen Anomalien in jeweils unterschiedlciher Stärke und Frequenz auf.



\section{Vorteile von RRCF} % Einleitung zu Bedarf des Datensatzes hier vor, oder ans Ende des Kapitels packen, oder in das nächste Kapitel so das dies ein reines Theorykapitel wird, oder Datensatz unabhängig beschreiben

RRCF wird zur Analyse des dieser Arbeit zugrunde legendem Datensatzes benutzt, da dass Verfahren eine Reihe von Vorteilen besitzt \cite{bartos2019rrcfImpl}:
\begin{itemize}
\item \textit{Anwendbarkeit auf Streaming-Daten}: Neue Datenpunkte können in die konstruierten Bäume eingegliedert werden ohne das diese neu aufgebaut werden müssen.
\item \textit{Geeignet für hoch dimensionale Daten}: Die angewandte Baumstruktur ist sehr geeignet für das aufnehmen von hochdimensionalen Daten. Da der Algorithmus zwischen wichtigen und unwichtigen Dimensionen unterscheiden kann, wird auch der Einfluss von solchen unwichtigen Dimensionen eingeschränkt.
\item \textit{Robust gegenüber Duplikaten}: Duplikate
\item \textit{Ausgabe in vorm einer Bewertung}: Eine Bewertende Ausgabe ist nützlich, da
\end{itemize} 